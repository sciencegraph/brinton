\documentclass[jss]{jss}
\usepackage[utf8]{inputenc}

\providecommand{\tightlist}{%
  \setlength{\itemsep}{0pt}\setlength{\parskip}{0pt}}

\author{
Pere Millán-Martínez\\UVic-UCC Servei Català de Trànsit \And Ramon Oller-Piqué\\UVic-UCC
}
\title{A Functional System for Statistical Graphics Presentation: \pkg{brinton}}

\Plainauthor{Pere Millán-Martínez, Ramon Oller-Piqué}
\Plaintitle{A Functional System for Statistical Graphics Presentation: \pkg{brinton}}
\Shorttitle{\pkg{brinton}: A Functional System for Statistical Graphics Presentation}

\Abstract{
Este artículo presenta el paquete \pkg{brinton} de \proglang{R}
desarrollado para el análisis gráfico exploratorio de datos y la
selección de una representación gráfica estadística en particular. Sobre
la base de \pkg{ggplot2} y \pkg{gridExtra} el paquete \pkg{brinton}
introduce las funciones \code{wideplot()} y \code{longplot()} que
presentan abanicos de gráficas estadísticas posibles a partir de una
selección de variables. Complementaria a estas dos funciones, también
introduce la función \code{showplot()} que permite seleccionar una
gráfica específica y ajustar sus propiedades. Este conjunto de funciones
se demuestra útil para entender la estructura de un conjunto rectangular
de datos, dejarse sorprender por propiedades no esperadas en los datos,
evaluar diferentes representaciones gráficas de éstos y seleccionar una
gráfica en particular.
}

\Keywords{Automated design, graphic design, statistical graphics, information visualization, exploratory data analysis}
\Plainkeywords{automated design, graphic design, statistical graphics, information visualization, exploratory data analysis}

%% publication information
%% \Volume{50}
%% \Issue{9}
%% \Month{June}
%% \Year{2012}
%% \Submitdate{}
%% \Acceptdate{2012-06-04}

\Address{
    Pere Millán-Martínez\\
  UVic-UCC Servei Català de Trànsit\\
  Carrer Diputació, 355 08009 Barcelona\\
  E-mail: \email{info@sciencegraph.org}\\
  URL: \url{http://sciencegraph.org}\\~\\
    }

% Pandoc header

\usepackage{amsmath}

\begin{document}

\hypertarget{introduction}{%
\section{Introduction}\label{introduction}}

En 1977 R.L. Oliver \citeyearpar{Oliver1977} vinculó la satisfacción a
la expectativa en su ``Expectation disconfirmation theory'' (EDT). J.W.
Tukey señaló que ``The greatest value of a picture is when it
\emph{forces} us to notice what we never expected to see''
\citep[p.iv]{Tukey1977} lo que atribuye un valor a la gráfica relativo a
la expectativa. En el campo del análisis exploratorio de datos la
expectativa juega un papel especial generalmente debido a su ausencia
porque las hipótesis no estan preestablecidas sinó que provienen de la
observación de los datos. Las hipótesis o la definición del problema,
como señaló J. Bertin ese mismo preciso año \citep[p.2]{Bertin1977}, no
son automatizables y aquí nace el reto de presentar gráficas a un
usuario para que éste observe los datos, pueda plantear una hipótesis y
afinar la selección de la gráfica estadística que dé respuesta a estas
hipótesis.

Puestos a elegir una estrategia para presentar gráficas a un usuario,
Millán-Martínez y Valero-Mora \citeyearpar{Millan2018}, por ejemplo,
diferencian las estrategias según si éstas se basan en las
características de los datos -- \emph{functional design}--
\citep[\citet{Gnanamgari1981}, \citet{Kamps1999},
\citet{ValeroMora2012}]{Benson1977}, en los hábitos de un usuario --
\emph{content-based filtering} --, en los hábitos de un grup de usuarios
-- \emph{collaborative filtering} -- \citep{Mutlu2015}, en las tareas a
realizar por los usuarios -- \emph{task design} --
\citep[\citet{Casner1991}]{Bowman1967}, en las características de la
percepción humana -- \emph{perceptual design} --
\citep[\citet{Mackinlay1986}]{Cleveland1984}, en las limitaciones del
canal de comunicación o la pantalla en la que se proyectan las gráficas
-- \emph{responsive design} -- \citep{Gnanamgari1981} y, finalmente, en
la selección de carácterísticas de la gráfica deseada o modelos de
representación -- \emph{this could be called model design} --
\citep{Roth1994}. Si, como hemos dicho anteriormente, carecemos de
hipótesis, si obviamos las limitaciones del hardware y el recuerdo en la
selección de gráficas realizado en otras ocasiones, queda la posibilidad
de acotar las gráficas a presentar según las características de los
datos, las características de la percepción humana y la selección de
modelos de representación.

En el entorno de programación \proglang{R}, hay implementadas diferentes
de las estrategias antes citadas. El \emph{functional design} se
encuentra, por ejemplo, implementada en la función \code{plot()} que, si
la aplicamos al dataset \texttt{cars} produce un diagrama de dispersión
(\emph{scatterplot}) dado que éste contiene dos variables numéricas y es
de classe \code{data.frame} mientras que, si lo aplicamos al dataset
\texttt{airmiles} produce un diagrama de línea (\emph{line graph})
porque éste contiene una única variable numérica y es de clase
\code{ts}. El \emph{task design} se encuentra implementado en multitud
de librerías como por ejemplo \pkg{survminer} \citep{Therneau2015} que
incluye la función \code{ggsurvplot()} para componer gráficas
específicas para el análisis de supervivencia (\emph{survival
analysis}). El \emph{model design} lo encontramos en funciones básicas
como por ejemplo \code{barplot()} que produce un diagrama de barras
(\emph{bar graph}), \code{hist()} que produce un
histograma(\emph{histogram}) o \code{pie()} que produce un diagramas de
pastel (\emph{pie charts}). El \emph{perceptual design} también se
encuentra implementado en librerías como por ejemplo \pkg{ggplot2}
\citep{Wickham2016} en aspectos como el tamaño, forma o el color de los
puntos que asigna por defecto, las líneas de ayuda (\emph{grid lines}) o
el color del fondo del panel (\emph{panel background color}).

Este artículo presenta el paquete \pkg{brinton} que explora una nueva
estrategia implementada sobre en el entorno de programación \proglang{R}
y las librerías \pkg{ggplot2} y \pkg{gridExtra} entre otras. Esta
estrategia consiste en presentar al usuario un abanico de gráficas
posibles a partir de las características de los datos para que, una vez
observados, el usuario pueda plantearse preguntas y explorar nuevos
catálogos de gráficas o una gráfica en particular.

\hypertarget{multipanel-graphics}{%
\section{Multipanel graphics}\label{multipanel-graphics}}

Existen diferentes tipos de gráficas multipanel según la variadad de
gráficas y el origen de los datos que éstas muestran. Por un lado
tenemos los cuadros de mando (\emph{dashboards}) que generalmente
combinan en un espacio limitado, diferentes tipos de gráficas en
diferentes paneles, cada una de las cuales puede representar datos de
diferentes orígines y cuya utilidad es monitorizar procesos complejos.
Por otro lado tenemos las gráficas condicionadas (\emph{conditioning
plots})\footnote{La terminología de las gráficas condicionadas no és
  unánime. Primero fueron primero descritas por J. Bertin como
  \emph{séries homogènes} \citep[p.?]{Bertin1967}, luego E.Tufte las dio
  a conocer como \emph{small multiples} \citep{Tufte1983}, W.S.
  Cleveland se refirió a ellas como \emph{juxtaposed panels}
  \citep[p.200]{Cleveland1985} y también como \emph{trellis graphics}
  \citep{Becker1996}, en el entorno de \proglang{R} de conocen
  básicamente como \emph{lattice graphics} \citep{Sarkar2008}.} que
muestran un mismo tipo de gráfica que se repite en diferentes paneles
que generalmente conservan la misma escala y que representan diferentes
subconjuntos de unos datos según una o dos variables. Otro ejemplo de
gráficas multipanel lo tenemos las matrices de gráficas (\emph{matrix of
plots}) que relacionan pares de variables de un conjunto de datos, como
por ejemplo las matrices de diagramas de dispersión (\emph{scatterplot
matrix}) \citep{Hartigan1975} o las matrices de gráficas de elipse HE
(\emph{HE plots}) \citep{Friendly2007}.

Existen básicamente tres vías para graphicar datos en \proglang{R}. La
primera es utilizar \emph{base graphics} y, si se pretende componer
gráficas multipanel

y por consiguiente cuesta solicitar a un sistema la presentación de una
gráfica u otra. Esto genera la conveniencia de utilizar métodos gráficos
que, por un lado, relacionen los valores de las diferentes variables y,
por otro lado, que estos métodos gráficos incluyan una variedad de tipos
de gráficas que permitan responder a diferentes preguntas que se puedan
formular.

\hypertarget{code-formatting}{%
\subsection{Code formatting}\label{code-formatting}}

Don't use markdown, instead use the more precise latex commands:

\begin{itemize}
\item
  \proglang{R}
\item
  \pkg{brinton}
\item
  \code{print("abc")}
\end{itemize}

\hypertarget{r-code}{%
\section{R code}\label{r-code}}

Can be inserted in regular R markdown blocks.

\begin{CodeChunk}

\begin{CodeInput}
R> x <- 1:10
R> x
\end{CodeInput}

\begin{CodeOutput}
 [1]  1  2  3  4  5  6  7  8  9 10
\end{CodeOutput}
\end{CodeChunk}

\bibliography{JSS.bib}


\end{document}

